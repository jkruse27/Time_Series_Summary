\documentclass[../main.tex]{subfiles}
\graphicspath{{\subfix{../images/}}}
\usepackage{subfiles}
\usepackage{subfig}

\begin{document}
    \subsection{Applications in Biosignal Analysis}
        When dealing with anomaly detection in medical and health monitoring applications, most of the times we are dealing with patient records. In this context, the anomalies present in the data can be due to several reasons, from indications of abnormal patient condition to instrumentation errors. As this project is related to time series analysis, anomaly detection in bio signals within this context will be the focus of analysis, but readers interested in research on other types of data can refer to \cite{fernando2021deep} for a review on deep learning approaches for general medical data and \cite{taboada2009images} for a review on anomaly detection in medical images. \par

        As was shown already, many datasets, especially when dealing with biosignals in smart wearables and such, are time series in nature.
\end{document}